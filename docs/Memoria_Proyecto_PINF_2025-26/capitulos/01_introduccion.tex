\section{Introducción}
El presente documento recoge la memoria del proyecto correspondiente a la asignatura Proyectos Informáticos (PINF) del curso académico 2025--26 en la Universidad de Cádiz. En él se describen las distintas fases del desarrollo, desde el análisis inicial hasta la implementación y las pruebas finales, incluyendo las decisiones de diseño y los resultados obtenidos.

\subsection{Objetivo del Proyecto}
El objetivo principal es modernizar y optimizar el sistema científico-técnico utilizado para la elaboración del libro Almanaque Náutico de la Armada, empleando tecnologías actuales que permitan mejorar la eficiencia en la generación y la distribución de contenidos. Se pretende desarrollar una solución que facilite las actualizaciones periódicas del almanaque y garantice la precisión y la vigencia de la información proporcionada a los usuarios finales.

\subsection{Estructura del Documento}
La memoria está organizada en varios capítulos que abordan las diferentes etapas del proyecto:
\begin{itemize}
    \item \textbf{Análisis:} Se detallan los requisitos funcionales y no funcionales, así como el estudio del sistema actual y las necesidades de los usuarios.
    \item \textbf{Diseño:} Se presentan las decisiones de diseño tomadas, incluyendo la arquitectura del sistema, la selección de tecnologías y la planificación del desarrollo.
    \item \textbf{Implementación:} Se describen las fases de desarrollo del software, las herramientas utilizadas y los desafíos encontrados durante la codificación.
    \item \textbf{Pruebas:} Se explican los métodos de prueba aplicados para asegurar la calidad del software, incluyendo pruebas unitarias, de integración y de aceptación.
    \item \textbf{Manuales:} Se incluyen tanto el manual de usuario como el manual técnico para facilitar el uso y mantenimiento del sistema desarrollado.
    \item \textbf{Planificación:} Se detalla la gestión del proyecto, incluyendo cronogramas, asignación de recursos y seguimiento del progreso.
    \item \textbf{Conclusiones:} Se resumen los logros alcanzados, las lecciones aprendidas y las posibles mejoras futuras.
    \item \textbf{Anexos:} Se proporcionan documentos adicionales relevantes para el proyecto, como diagramas, códigos fuente y referencias bibliográficas.
\end{itemize}
Cada capítulo está diseñado para proporcionar una visión clara y detallada de cada fase del proyecto, facilitando la comprensión del proceso seguido y los resultados obtenidos.

\subsection{Alcance del Proyecto} REVISAR TEXTO
El alcance del proyecto abarca la modernización completa del sistema de elaboración del Almanaque Náutico de la Armada, incluyendo la digitalización de procesos, la implementación de nuevas tecnologías para la gestión de datos y la creación de una plataforma accesible para los usuarios finales. Se pretende que el sistema sea escalable y adaptable a futuras necesidades, permitiendo la incorporación de nuevas funcionalidades y mejoras conforme evolucione el entorno tecnológico y las demandas de los usuarios.

\subsection{Metodología de Trabajo}  REVISAR TEXTO
Para llevar a cabo el proyecto, se ha seguido una metodología ágil que permite una adaptación continua a los cambios y una mejora constante del producto final. Se han realizado reuniones periódicas con los stakeholders para asegurar que los requisitos se cumplen y se han utilizado herramientas de gestión de proyectos para mantener un seguimiento detallado del progreso y las tareas pendientes. La colaboración entre los miembros del equipo ha sido fundamental para el éxito del proyecto, fomentando la comunicación abierta y la resolución conjunta de problemas.

\subsection{Tecnologías Utilizadas} REVISAR TEXTO
El desarrollo del proyecto ha implicado la utilización de diversas tecnologías modernas que facilitan la creación de un sistema robusto y eficiente. Entre las principales tecnologías empleadas se encuentran:
% Citar las tecnologías específicas utilizadas en el proyecto, por ejemplo:
% \begin{itemize}
%     \item Lenguajes de programación: Python, JavaScript
%     \item Frameworks: Django, React
%     \item Herramientas de control de versiones: Git, GitHub, Fork (para gestión local)
%     \item Plataformas de despliegue: Docker
%     \item etc.
% \end{itemize}
Estas tecnologías han sido seleccionadas por su capacidad para cumplir con los requisitos del proyecto, su facilidad de uso y su amplia comunidad de soporte.

\subsection{Estructura del Equipo}
REVISAR TEXTO
\begin{itemize}
    \item \textbf{Gestor de Proyecto:} Responsable de la planificación, seguimiento y coordinación del equipo.
    \item \textbf{Analista de Requisitos:} Encargado de recopilar y documentar los requisitos del sistema.
    \item \textbf{Diseñador de Software:} Responsable del diseño arquitectónico y la selección de tecnologías.
    \item \textbf{Desarrolladores:} Encargados de la codificación y la implementación del sistema.
    \item \textbf{Tester/QA:} Responsable de las pruebas y aseguramiento de la calidad del software.
    \item \textbf{Documentalista:} Encargado de la elaboración de la documentación técnica y de usuario.
\end{itemize}

\begin{table}[htbp]
    \centering
    \begin{tabular}{@{} l l @{}}
        \textbf{Administrador:}  & Alberto Gómez Moreno      \\
        \textbf{Analista:}       & Juan Cabañas Carbonell    \\
        \textbf{Diseñador:}      & Alberto Gómez Moreno      \\
        \textbf{Programadores:}  & Alberto Gómez Moreno      \\
                                 & Alberto Periñán Dávila    \\
                                 & Raúl Silva Bienvenido     \\
                                 & José Carlos Leal Iglesias \\
                                 & Sergio Cabrera Marín      \\
                                 & Carlos Fernández Cabeza   \\
        \textbf{Tester:}         & Juan Cabañas Carbonell    \\
        \textbf{Documentalista:} & Carlos Fernández Cabeza   \\
    \end{tabular}
\end{table}