\section{Manual Técnico}
INCOMPLETO El manual técnico tiene como objetivo proporcionar una guía detallada sobre la arquitectura y el funcionamiento interno del sistema desarrollado en el Proyecto PINF 2025--26. Este manual está dirigido a los desarrolladores y personal técnico, y debe ser lo suficientemente detallado como para permitirles comprender y mantener el sistema de manera efectiva.

\subsection{Arquitectura del Sistema}
La arquitectura del sistema se basa en una estructura modular que permite una fácil escalabilidad y mantenimiento. A continuación se describen los principales componentes del sistema y sus responsabilidades (EJEMPLO):
\begin{itemize}
    \item \textbf{Componente 1:} Descripción del componente 1.
    \item \textbf{Componente 2:} Descripción del componente 2.
    \item \textbf{Componente 3:} Descripción del componente 3.
\end{itemize}

\subsection{Tecnologías Utilizadas}
El sistema se ha desarrollado utilizando las siguientes tecnologías (EJEMPLO):
\begin{itemize}
    \item \textbf{Lenguaje de Programación:} Descripción del lenguaje de programación utilizado.
    \item \textbf{Framework:} Descripción del framework utilizado.
    \item \textbf{Base de Datos:} Descripción de la base de datos utilizada.
\end{itemize}

\subsection{Guía de Mantenimiento}
Para garantizar el correcto funcionamiento del sistema a lo largo del tiempo, se recomienda seguir las siguientes pautas de mantenimiento (EJEMPLO):
\begin{itemize}
    \item Realizar copias de seguridad periódicas de la base de datos.
    \item Monitorizar el rendimiento del sistema y realizar ajustes según sea necesario.
    \item Mantener actualizadas las dependencias y bibliotecas utilizadas en el desarrollo.
\end{itemize}
