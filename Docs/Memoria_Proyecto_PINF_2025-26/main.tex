\documentclass[12pt,a4paper]{article}
\usepackage[spanish]{babel}
\usepackage[utf8]{inputenc}
\usepackage[T1]{fontenc}
\usepackage{graphicx}
\usepackage{csquotes}
\usepackage{booktabs}
\usepackage{geometry}
\usepackage{amsmath, amssymb}
\usepackage{float}
\usepackage{caption}
\usepackage{subcaption}
\usepackage{enumitem}
\usepackage{hyperref}
\usepackage{fancyhdr}
\usepackage{pdfpages}
\usepackage[backend=biber]{biblatex} % backend recomendado para biblatex
\addbibresource{referencias.bib}
\geometry{margin=2.5cm}
% Comandos personalizados para facilitar escritura
\newcommand{\app}[1]{\texttt{#1}} % para nombres de aplicaciones
\newcommand{\cod}[1]{\texttt{#1}} % para fragmentos de código
\newcommand{\img}[2]{\begin{figure}[H]\centering\includegraphics[width=0.8\textwidth]{#1}\caption{#2}\end{figure}}

\usepackage{estilos/formato_pinf}

\pagestyle{fancy}
\fancyhead[L]{Proyecto PINF 2025–26}
\fancyhead[R]{Universidad de Cádiz}
\fancyfoot[C]{\thepage}

\setlength{\headheight}{15pt} % evita el aviso de fancyhdr
% \addtolength{\topmargin}{-2.5pt} % opcional si persiste el aviso

\begin{document}

\begin{titlepage}
\centering

\vspace*{0.5cm}

{\Large\scshape Universidad de Cádiz\par}
\vspace{0.25cm}
{\large Grado en Ingeniería Informática\par}
\vspace{0.15cm}
{\large Curso 2025/2026\par}

\vspace{1.8cm}
{\Huge\bfseries Memoria de PINF\par}

\vspace{0.4cm}
\rule{0.6\textwidth}{0.4pt}\par

\vspace{0.8cm}
{\Large Modernización del Sistema para la\\[0.2cm]
Elaboración del Almanaque Náutico de la Armada\par}

\vspace{2.5cm}

{\large\bfseries Integrantes\par}
\vspace{0.4cm}

{\normalsize
Alberto Gómez Moreno\\
Juan María Cabañas Carbonell\\
Alberto Periñán Dávila\\
Raúl Silva Bienvenido\\
Sergio Cabrera Marín\\
José Carlos Leal Iglesias\\
Carlos Fernández Cabeza\par
}

\vfill

{\large Cádiz, \today\par}

\end{titlepage}

\newpage
\tableofcontents
\newpage

\section{Introducción}
El presente documento recoge la memoria del proyecto correspondiente a la asignatura Proyectos Informáticos (PINF) del curso académico 2025--26 en la Universidad de Cádiz. En él se describen las distintas fases del desarrollo, desde el análisis inicial hasta la implementación y las pruebas finales, incluyendo las decisiones de diseño y los resultados obtenidos.

\subsection{Objetivo del Proyecto}
El objetivo principal es modernizar y optimizar el sistema científico-técnico utilizado para la elaboración del libro Almanaque Náutico de la Armada, empleando tecnologías actuales que permitan mejorar la eficiencia en la generación y la distribución de contenidos. Se pretende desarrollar una solución que facilite las actualizaciones periódicas del almanaque y garantice la precisión y la vigencia de la información proporcionada a los usuarios finales.

\subsection{Estructura del Documento}
La memoria está organizada en varios capítulos que abordan las diferentes etapas del proyecto:
\begin{itemize}
    \item \textbf{Análisis:} Se detallan los requisitos funcionales y no funcionales, así como el estudio del sistema actual y las necesidades de los usuarios.
    \item \textbf{Diseño:} Se presentan las decisiones de diseño tomadas, incluyendo la arquitectura del sistema, la selección de tecnologías y la planificación del desarrollo.
    \item \textbf{Implementación:} Se describen las fases de desarrollo del software, las herramientas utilizadas y los desafíos encontrados durante la codificación.
    \item \textbf{Pruebas:} Se explican los métodos de prueba aplicados para asegurar la calidad del software, incluyendo pruebas unitarias, de integración y de aceptación.
    \item \textbf{Manuales:} Se incluyen tanto el manual de usuario como el manual técnico para facilitar el uso y mantenimiento del sistema desarrollado.
    \item \textbf{Planificación:} Se detalla la gestión del proyecto, incluyendo cronogramas, asignación de recursos y seguimiento del progreso.
    \item \textbf{Conclusiones:} Se resumen los logros alcanzados, las lecciones aprendidas y las posibles mejoras futuras.
    \item \textbf{Anexos:} Se proporcionan documentos adicionales relevantes para el proyecto, como diagramas, códigos fuente y referencias bibliográficas.
\end{itemize}
Cada capítulo está diseñado para proporcionar una visión clara y detallada de cada fase del proyecto, facilitando la comprensión del proceso seguido y los resultados obtenidos.

\subsection{Alcance del Proyecto} REVISAR TEXTO
El alcance del proyecto abarca la modernización completa del sistema de elaboración del Almanaque Náutico de la Armada, incluyendo la digitalización de procesos, la implementación de nuevas tecnologías para la gestión de datos y la creación de una plataforma accesible para los usuarios finales. Se pretende que el sistema sea escalable y adaptable a futuras necesidades, permitiendo la incorporación de nuevas funcionalidades y mejoras conforme evolucione el entorno tecnológico y las demandas de los usuarios.

\subsection{Metodología de Trabajo}  REVISAR TEXTO
Para llevar a cabo el proyecto, se ha seguido una metodología ágil que permite una adaptación continua a los cambios y una mejora constante del producto final. Se han realizado reuniones periódicas con los stakeholders para asegurar que los requisitos se cumplen y se han utilizado herramientas de gestión de proyectos para mantener un seguimiento detallado del progreso y las tareas pendientes. La colaboración entre los miembros del equipo ha sido fundamental para el éxito del proyecto, fomentando la comunicación abierta y la resolución conjunta de problemas.

\subsection{Tecnologías Utilizadas} REVISAR TEXTO
El desarrollo del proyecto ha implicado la utilización de diversas tecnologías modernas que facilitan la creación de un sistema robusto y eficiente. Entre las principales tecnologías empleadas se encuentran:
% Citar las tecnologías específicas utilizadas en el proyecto, por ejemplo:
% \begin{itemize}
%     \item Lenguajes de programación: Python, JavaScript
%     \item Frameworks: Django, React
%     \item Herramientas de control de versiones: Git, GitHub, Fork (para gestión local)
%     \item Plataformas de despliegue: Docker
%     \item etc.
% \end{itemize}
Estas tecnologías han sido seleccionadas por su capacidad para cumplir con los requisitos del proyecto, su facilidad de uso y su amplia comunidad de soporte.

\subsection{Estructura del Equipo}
REVISAR TEXTO
\begin{itemize}
    \item \textbf{Gestor de Proyecto:} Responsable de la planificación, seguimiento y coordinación del equipo.
    \item \textbf{Analista de Requisitos:} Encargado de recopilar y documentar los requisitos del sistema.
    \item \textbf{Diseñador de Software:} Responsable del diseño arquitectónico y la selección de tecnologías.
    \item \textbf{Desarrolladores:} Encargados de la codificación y la implementación del sistema.
    \item \textbf{Tester/QA:} Responsable de las pruebas y aseguramiento de la calidad del software.
    \item \textbf{Documentalista:} Encargado de la elaboración de la documentación técnica y de usuario.
\end{itemize}

\begin{table}[htbp]
    \centering
    \begin{tabular}{@{} l l @{}}
        \textbf{Administrador:}  & Alberto Gómez Moreno      \\
        \textbf{Analista:}       & Juan Cabañas Carbonell    \\
        \textbf{Diseñador:}      & Alberto Gómez Moreno      \\
        \textbf{Programadores:}  & Alberto Gómez Moreno      \\
                                 & Alberto Periñán Dávila    \\
                                 & Raúl Silva Bienvenido     \\
                                 & José Carlos Leal Iglesias \\
                                 & Sergio Cabrera Marín      \\
                                 & Carlos Fernández Cabeza   \\
        \textbf{Tester:}         & Juan Cabañas Carbonell    \\
        \textbf{Documentalista:} & Carlos Fernández Cabeza   \\
    \end{tabular}
\end{table}
\section{Análisis}
INCOMPLETO En esta sección se llevará a cabo un análisis detallado del problema a resolver, así como de los requisitos del sistema. Se utilizarán técnicas de análisis de requisitos y se elaborarán diagramas que faciliten la comprensión del sistema.

\section{Diseño}
En esta sección se presentará el diseño del sistema, incluyendo la arquitectura general, los componentes principales y las interacciones entre ellos. Se utilizarán diagramas de diseño para ilustrar la estructura del sistema y facilitar su comprensión.

\section{Implementación}
En esta sección se describirá la implementación del sistema, incluyendo las tecnologías utilizadas, la estructura del código y los principales desafíos encontrados durante el desarrollo.

\section{Pruebas}
INCOMPLETO En esta sección se detallarán las pruebas realizadas al sistema, incluyendo las pruebas unitarias, de integración y de aceptación. Se presentarán los resultados obtenidos y se analizarán los posibles errores encontrados.

\setlength{\headheight}{15pt}

\section{Manual de Usuario}
INCOMPLETO En esta sección se proporcionará un manual de usuario detallado para el sistema desarrollado. Se incluirán instrucciones sobre la instalación, configuración y uso del sistema, así como la resolución de problemas comunes que puedan surgir durante su utilización.

\subsection{Instalación}
Para instalar el sistema, siga los siguientes pasos (EJEMPLO):
\begin{enumerate}
    \item Descargue el paquete de instalación desde el sitio web oficial.
    \item Ejecute el instalador y siga las instrucciones en pantalla.
    \item Una vez completada la instalación, inicie la aplicación.
\end{enumerate}

\subsection{Uso del Sistema}
Una vez que el sistema esté instalado, los usuarios pueden comenzar a utilizarlo. A continuación se presentan las principales funcionalidades del sistema (EJEMPLO):
\begin{itemize}
    \item \textbf{Funcionalidad 1:} Descripción de la funcionalidad 1.
    \item \textbf{Funcionalidad 2:} Descripción de la funcionalidad 2.
    \item \textbf{Funcionalidad 3:} Descripción de la funcionalidad 3.
\end{itemize}

\subsection{Resolución de Problemas}
En caso de encontrar problemas al utilizar el sistema, se recomienda seguir los siguientes pasos (EJEMPLO):
\begin{itemize}
    \item Verifique que tiene instalada la última versión del sistema.
    \item Consulte la sección de preguntas frecuentes en el sitio web oficial.
    \item Si el problema persiste, contacte con el soporte técnico.
\end{itemize}

\section{Manual Técnico}
INCOMPLETO El manual técnico tiene como objetivo proporcionar una guía detallada sobre la arquitectura y el funcionamiento interno del sistema desarrollado en el Proyecto PINF 2025--26. Este manual está dirigido a los desarrolladores y personal técnico, y debe ser lo suficientemente detallado como para permitirles comprender y mantener el sistema de manera efectiva.

\subsection{Arquitectura del Sistema}
La arquitectura del sistema se basa en una estructura modular que permite una fácil escalabilidad y mantenimiento. A continuación se describen los principales componentes del sistema y sus responsabilidades (EJEMPLO):
\begin{itemize}
    \item \textbf{Componente 1:} Descripción del componente 1.
    \item \textbf{Componente 2:} Descripción del componente 2.
    \item \textbf{Componente 3:} Descripción del componente 3.
\end{itemize}

\subsection{Tecnologías Utilizadas}
El sistema se ha desarrollado utilizando las siguientes tecnologías (EJEMPLO):
\begin{itemize}
    \item \textbf{Lenguaje de Programación:} Descripción del lenguaje de programación utilizado.
    \item \textbf{Framework:} Descripción del framework utilizado.
    \item \textbf{Base de Datos:} Descripción de la base de datos utilizada.
\end{itemize}

\subsection{Guía de Mantenimiento}
Para garantizar el correcto funcionamiento del sistema a lo largo del tiempo, se recomienda seguir las siguientes pautas de mantenimiento (EJEMPLO):
\begin{itemize}
    \item Realizar copias de seguridad periódicas de la base de datos.
    \item Monitorizar el rendimiento del sistema y realizar ajustes según sea necesario.
    \item Mantener actualizadas las dependencias y bibliotecas utilizadas en el desarrollo.
\end{itemize}

\section{Planificación}
En esta sección se describirá la planificación del proyecto, incluyendo los objetivos, el alcance y los plazos establecidos. Se presentarán las principales actividades realizadas durante la fase de planificación y se analizarán los resultados obtenidos.

\section{Conclusiones}
INCOMPLETO En esta sección se presentarán las conclusiones del proyecto, incluyendo los logros alcanzados, las lecciones aprendidas y las recomendaciones para futuros trabajos.

\section{Anexos}
INCOMPLETO En esta sección se incluirán los anexos relevantes para el proyecto, como documentos adicionales, diagramas, código fuente y cualquier otro material que pueda ser útil para comprender mejor el trabajo realizado.


\nocite{*} % imprime todas las entradas del .bib aunque no estén citadas
\printbibliography%

\end{document}
